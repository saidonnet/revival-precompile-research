\documentclass{article}
\usepackage[utf8]{inputenc}
\usepackage{geometry}
\geometry{margin=0.7in}
\usepackage{listings}
\usepackage{xcolor}
\usepackage{hyperref}
\usepackage{natbib}
\usepackage{amsmath}
\usepackage{amsfonts}
\usepackage{amssymb}

\lstset{
  basicstyle=\ttfamily\footnotesize,
  breaklines=true,
  frame=single,
  numbers=left,
  numberstyle=\tiny\color{gray},
  captionpos=b,
  showstringspaces=false
}

\sloppy

\title{PARADISE: Proactive State Expiry via Adaptive Incentives on EVM-Compatible Proof-of-Stake Chains\thanks{This work was developed using adversarial AI synthesis methodology. Complete research timeline: 6 days. Total cost: <\$140. Full methodology and code: https://github.com/saidonnet/revival-precompile-research}}

\author{
  Saïd RAHMANI \\
  Independent Researcher \\
  \texttt{saidonnet@gmail.com}
}

\date{\today}

\begin{document}

\maketitle

\begin{abstract}
This paper presents PARADISE (Proactive, Adaptive, Reputation-based, Incentive-driven State Expiry), a comprehensive solution to blockchain state bloat on EVM-compatible proof-of-stake networks. Unbounded state growth threatens long-term decentralization by increasing hardware requirements for validators, creating centralization pressure and barriers to entry. Our proposed framework integrates four key components: (1) a dynamic state lifecycle management system with four stages (Active, Inactive, Expired, Archived), (2) a cryptoeconomically secured stateless execution model powered by a decentralized Continuous Proof Market (CPM), (3) a transition to Verkle Trees~\cite{kuszmaul2018,buterin2021verkle} as the core state data structure, and (4) a permanent, verifiable archival layer. We provide a complete technical specification, reference implementation code, phased migration strategy, and rigorous security analysis. The PARADISE framework is designed to reduce state growth rate by over 90\%, enable new nodes to sync in under one hour, and preserve decentralization while maintaining backward compatibility. Our enhanced design addresses critical security vulnerabilities through reentrancy protection, nonce-based witness validation, anti-Sybil mechanisms, and MEV-resistant auction systems.
\end{abstract}

\section{Introduction}

The active state of EVM-compatible blockchains, comprising all account balances, contract code, and storage, is experiencing unsustainable growth that threatens the fundamental principles of decentralization. Current Ethereum mainnet state exceeds 1.2 TB with the active state trie approaching 200 GB, growing by approximately 50-80 GB per year~\cite{konstantopoulos2024stategrowth}. This state bloat imposes prohibitive hardware costs on validators, requiring high-end NVMe SSDs and increasingly large amounts of RAM (64GB+ recommended) for performant operation.

The bootstrap problem for new nodes is acute, with full synchronization from genesis taking days or weeks. This barrier to entry concentrates power among large staking providers and cloud services, undermining network censorship resistance and resilience. At current growth rates, running a full validating node will soon be economically prohibitive for individuals using consumer-grade hardware, creating a feedback loop of rising costs, validator consolidation, and centralization.

Existing approaches to state management, including various state rent proposals~\cite{buterin2021statesize}, have faced significant challenges in implementation due to complexity, backward compatibility concerns, and economic sustainability issues. The Ethereum roadmap recognizes statelessness as a critical component for long-term scalability~\cite{ethereumorg2025statelessness}, but comprehensive solutions integrating state expiry, stateless execution, and permanent archival have remained elusive.

This paper introduces the PARADISE framework, a holistic approach that addresses state bloat through coordinated mechanisms for state lifecycle management, stateless execution via cryptographic witnesses, advanced data structures, and decentralized archival. Our solution maintains backward compatibility while creating sustainable economic incentives that align storage costs with actual usage patterns.

\section{Background and Related Work}

\subsection{State Growth Problem}

The fundamental issue stems from the economic model where storage costs are paid once upfront through gas fees but impose perpetual costs on all validators. Smart contracts create storage without economic disincentive for long-term retention, and no mechanism exists to archive or expire unused state. This creates a tragedy of the commons where individual rational behavior leads to collective irrationality.

Recent analysis shows that Ethereum's state growth has accelerated, with the active state requiring increasingly sophisticated hardware for efficient access~\cite{konstantopoulos2024stategrowth}. The memory requirements exclude consumer hardware, creating centralization pressure as only datacenters can economically validate the network.

\subsection{Stateless Execution}

Stateless execution allows validators to verify blocks without holding the full state by using cryptographic witnesses that prove the validity of state transitions~\cite{ethereumorg2025statelessness}. The primary challenge is witness generation and distribution, as current Merkle Patricia Trie structures produce prohibitively large proofs (3-5KB per state access).

\subsection{Verkle Trees}

Verkle Trees represent a significant advancement in cryptographic data structures for blockchain state management~\cite{kuszmaul2018,buterin2021verkle}. Unlike traditional Merkle trees that require logarithmic proof sizes, Verkle Trees use polynomial commitments to achieve constant-size proofs of approximately 150 bytes per state access, representing a 20x improvement over current structures.

Two main polynomial commitment schemes exist: KZG commitments~\cite{kzg2010} offer constant-size proofs but require a trusted setup ceremony, while Inner Product Arguments (IPA)~\cite{bunz2018bulletproofs} provide logarithmic-sized proofs without trusted setup requirements. The long-term security implications favor IPA-based implementations despite slightly larger proof sizes.

\subsection{Decentralized Storage}

Permanent data preservation requires sustainable economic models. Arweave's endowment model~\cite{williams2018arweave} provides perpetual storage through one-time payments, while Filecoin's proof-of-spacetime mechanism~\cite{protocollabs2017filecoin} ensures ongoing storage verification through cryptographic proofs.

\section{PARADISE Framework Architecture}

\subsection{State Lifecycle Management}

We introduce a formal four-stage state lifecycle managed by protocol rules:

\begin{enumerate}
\item \textbf{Active:} Default state held by all validators in the hot state tree
\item \textbf{Inactive:} State marked after 365 days without read/write access
\item \textbf{Expired:} Inactive state pruned from active Verkle tree during periodic expiry epochs
\item \textbf{Archived:} Pruned state stored permanently in decentralized archival layer
\end{enumerate}

\subsubsection{State Revival Mechanism}

To maintain composability, expired state can be revived on-demand through a \texttt{RevivalPrecompile} contract. When transactions access expired state, the EVM automatically triggers revival using provided Verkle proofs. This process requires witnesses demonstrating the last known state values before expiry, sourced from the archival layer.

The revival mechanism handles cross-contract calls transparently. When active Contract A calls expired Contract B, the transaction must include witnesses for Contract B's state. The EVM detects the expired state and automatically triggers revival, with gas costs paid by the transaction originator.

\subsection{Stateless Execution and Continuous Proof Market}

Stateless execution enables validators to verify blocks using only block data and accompanying witnesses, without maintaining full state. We address the critical challenge of witness generation through the Continuous Proof Market (CPM), a decentralized marketplace for cryptographic proofs.

\subsubsection{Node Specialization}

The network evolves to support specialized roles:

\begin{itemize}
\item \textbf{Stateless Validators:} Default validator type holding no active state, verifying blocks through provided witnesses
\item \textbf{Provers:} Specialized staked nodes maintaining full state, generating Verkle proofs for mempool transactions
\item \textbf{Archival Nodes:} Nodes storing pruned state permanently in decentralized backends, serving revival proofs
\end{itemize}

\subsubsection{Continuous Proof Market Design}

The CPM operates as a decentralized off-chain order book with on-chain settlement. Provers monitor the mempool and speculatively generate witnesses for pending transactions, posting ask orders in the marketplace. Block proposers create bid orders for required witnesses, with a matching engine connecting buyers and sellers.

Security is ensured through a multi-dimensional reputation system, slashable bonds in the \texttt{ProverRegistry}, and on-chain adjudication contracts that punish misbehavior including proof withholding and forgery.

\subsection{Verkle Tree Implementation}

We advocate for IPA-based Verkle Trees to avoid trusted setup risks associated with KZG commitments. The migration follows a phased approach using dual-tree transition:

\begin{enumerate}
\item \textbf{Hardfork 1:} Add Verkle tree root alongside existing MPT root
\item \textbf{Transition Period:} Copy MPT state to Verkle tree on access, new writes go directly to Verkle tree
\item \textbf{Hardfork 2:} Remove MPT root, operate exclusively on Verkle tree
\end{enumerate}

\subsection{Decentralized Archival Layer}

Expired state is preserved through a network of incentivized archival nodes using hybrid storage backends. Primary storage utilizes Arweave's permanent endowment model, with redundancy provided by Filecoin or custom DHT networks. Economic incentives are funded through a portion of state revival fees, with proof-of-storage mechanisms ensuring data integrity.

\section{Enhanced Security Design}

Critical analysis reveals several security vulnerabilities in the initial design that must be addressed for production deployment.

\subsection{Revival Precompile Security}

The initial \texttt{RevivalPrecompile} design contains severe security flaws including reentrancy risks and replay attack vulnerabilities. Our enhanced implementation incorporates comprehensive security measures:

\begin{lstlisting}[language=C,caption={Enhanced Revival Precompile with Security Protections},label={lst:revival}]
contract RevivalPrecompileV2 {
    mapping(bytes32 => bool) private _revivalInProgress;
    mapping(bytes32 => uint256) private _witnessNonces;
    
    modifier nonReentrant(bytes32 key) {
        require(!_revivalInProgress[key], "Revival in progress");
        _revivalInProgress[key] = true;
        _;
        _revivalInProgress[key] = false;
    }
    
    function revive(bytes[] calldata witnesses, uint256[] calldata nonces) 
        external payable nonReentrant(keccak256(abi.encode(witnesses))) {
        require(witnesses.length == nonces.length, "Length mismatch");
        
        uint256 totalFee = 0;
        for (uint i = 0; i < witnesses.length; i++) {
            bytes32 witnessHash = keccak256(witnesses[i]);
            require(nonces[i] > _witnessNonces[witnessHash], "Stale witness");
            _witnessNonces[witnessHash] = nonces[i];
            
            (bool valid, uint256 fee) = _verifyAndPriceWitness(witnesses[i]);
            require(valid, "Invalid witness");
            totalFee += fee;
        }
        
        require(msg.value >= totalFee, "Insufficient fee");
        _batchRevive(witnesses);
        
        if (msg.value > totalFee) {
            payable(msg.sender).transfer(msg.value - totalFee);
        }
    }
}
\end{lstlisting}

\subsection{Anti-Sybil Mechanisms}

The \texttt{ProverRegistry} implements robust anti-Sybil measures including proof-of-unique-hardware and quadratic staking to prevent centralization:

\begin{lstlisting}[language=C,caption={Enhanced Prover Registry with Anti-Sybil Protection},label={lst:registry}]
contract ProverRegistryV2 {
    struct ProverV2 {
        address owner;
        uint256 stake;
        uint256 lastUpdated;
        string p2pEndpoint;
        ProverStatus status;
        ReputationV2 reputation;
        bytes32 hardwareFingerprint;
        uint256 geographicRegion;
    }
    
    function calculateMinStake(address prover) public view returns (uint256) {
        uint256 proverCount = getActiveProverCount();
        uint256 baseStake = MIN_STAKE;
        
        uint256 existingStake = provers[prover].stake;
        uint256 totalStake = getTotalStake();
        
        if (totalStake > 0) {
            uint256 concentration = (existingStake * 100) / totalStake;
            if (concentration > 5) {
                baseStake = baseStake * (concentration * concentration) / 100;
            }
        }
        
        return baseStake;
    }
    
    function registerWithProofOfUniqueHardware(
        string memory p2pEndpoint,
        bytes32 hardwareFingerprint,
        bytes memory hardwareProof,
        uint256 geographicRegion
    ) external payable {
        require(msg.value >= calculateMinStake(msg.sender), "Insufficient stake");
        require(_verifyHardwareProof(hardwareFingerprint, hardwareProof), "Invalid hardware proof");
        require(!_isHardwareFingerprintUsed(hardwareFingerprint), "Hardware already registered");
        
        // Registration logic continues...
    }
}
\end{lstlisting}

\subsection{MEV-Resistant Proof Marketplace}

To prevent MEV extraction and ensure fair price discovery, we implement a commit-reveal auction system:

\begin{lstlisting}[language=C,caption={MEV-Resistant Proof Marketplace},label={lst:marketplace}]
contract ProofMarketplaceV2 {
    struct AuctionV2 {
        bytes32 bundleHash;
        uint256 subsetCount;
        uint256 commitDeadline;
        uint256 revealDeadline;
        uint256 minReputation;
        bool finalized;
        mapping(uint256 => CommitRevealBid[]) bids;
    }
    
    struct CommitRevealBid {
        address bidder;
        bytes32 commitment;
        uint256 revealedAmount;
        bytes reputationProof;
        bool revealed;
        bool valid;
    }
    
    function commitBid(
        bytes32 bundleHash,
        uint256 subsetIndex,
        bytes32 commitment
    ) external payable {
        AuctionV2 storage auction = auctions[bundleHash];
        require(block.timestamp < auction.commitDeadline, "Commit phase ended");
        require(msg.value >= MIN_BID_BOND, "Insufficient bond");
        
        auction.bids[subsetIndex].push(CommitRevealBid({
            bidder: msg.sender,
            commitment: commitment,
            revealedAmount: 0,
            reputationProof: "",
            revealed: false,
            valid: false
        }));
    }
    
    function revealBid(
        bytes32 bundleHash,
        uint256 subsetIndex,
        uint256 bidIndex,
        uint256 amount,
        uint256 salt,
        bytes memory reputationProof
    ) external {
        // Reveal logic with cryptographic verification
        bytes32 expectedCommitment = keccak256(abi.encodePacked(amount, salt, reputationProof));
        require(bid.commitment == expectedCommitment, "Invalid reveal");
        
        // Additional verification and scoring logic...
    }
}
\end{lstlisting}

\section{Performance Optimizations}

\subsection{Batch Processing and Caching}

The enhanced Verkle Tree implementation incorporates sophisticated caching and batch processing mechanisms:

\begin{lstlisting}[language=C++,caption={Optimized Verkle Tree with Caching},label={lst:verkle}]
pub struct VerkleTreeV2<E: PairingEngine> {
    root: VerkleNode<E>,
    cache: LRUCache<Vec<u8>, VerkleNode<E>>,
    commitment_cache: LRUCache<Vec<u8>, E::G1Affine>,
}

impl<E: PairingEngine> VerkleTreeV2<E> {
    pub fn batch_update(&mut self, updates: Vec<(Vec<u8>, E::Fr)>) -> Result<E::G1Affine, Error> {
        let mut sorted_updates = updates;
        sorted_updates.sort_by(|a, b| a.0.cmp(&b.0));
        
        let mut affected_paths = HashSet::new();
        
        for (path, value) in sorted_updates {
            self._update_leaf(&path, value)?;
            
            for i in 0..path.len() {
                affected_paths.insert(path[..i].to_vec());
            }
        }
        
        let mut sorted_paths: Vec<_> = affected_paths.into_iter().collect();
        sorted_paths.sort_by(|a, b| b.len().cmp(&a.len()));
        
        for path in sorted_paths {
            self._recompute_commitment_cached(&path)?;
        }
        
        Ok(self.root.commitment)
    }
    
    fn _recompute_commitment_cached(&mut self, path: &[u8]) -> Result<(), Error> {
        if let Some(cached) = self.commitment_cache.get(path) {
            return Ok(());
        }
        
        let node = self._get_node_mut(path)?;
        let new_commitment = self._compute_polynomial_commitment(&node.children)?;
        node.commitment = new_commitment;
        
        self.commitment_cache.put(path.to_vec(), new_commitment);
        Ok(())
    }
}
\end{lstlisting}

\section{Migration Strategy}

The transition to PARADISE occurs through four carefully orchestrated hardforks:

\subsection{Hardfork I: Verkle Genesis}
\begin{itemize}
\item Add \texttt{verkle\_root} field to block headers
\item Implement dual-write mode for state updates
\item Begin copying MPT state to Verkle tree on access
\end{itemize}

\subsection{Hardfork II: Market Activation}
\begin{itemize}
\item Deploy \texttt{RevivalPrecompile}, \texttt{ProverRegistry}, and \texttt{ProofMarketplace}
\item Activate state expiry mechanism for old state
\item Introduce new gas economics reflecting witness costs
\end{itemize}

\subsection{Hardfork III: Stateless Mandate}
\begin{itemize}
\item Require witnesses for all state-accessing transactions
\item Enable stateless validator operation
\item Full CPM activation with slashing mechanisms
\end{itemize}

\subsection{Hardfork IV: MPT Sunset}
\begin{itemize}
\item Remove MPT root from block headers
\item Complete transition to Verkle-only operation
\item Enable pruning of legacy MPT data
\end{itemize}

\section{Security Analysis}

\subsection{Attack Vectors and Mitigations}

We identify and address several critical attack vectors:

\begin{itemize}
\item \textbf{Witness Withholding:} Mitigated through economic disincentives, slashing mechanisms, and marketplace redundancy
\item \textbf{Griefing Attacks:} Prevented through bonding requirements and reputation-based rate limiting
\item \textbf{Prover Centralization:} Addressed via quadratic staking, hardware fingerprinting, and geographic distribution incentives
\item \textbf{Archival Censorship:} Countered through storage redundancy and cryptographic verifiability
\end{itemize}

\subsection{Economic Security}

The framework incorporates dynamic fee markets similar to EIP-1559~\cite{eip1559} for revival costs, ensuring economic sustainability while preventing fee manipulation. Multi-dimensional reputation scoring and commit-reveal auctions provide robust protection against market manipulation and MEV extraction.

\section{Evaluation and Expected Impact}

The PARADISE framework is designed to achieve significant improvements in blockchain sustainability:

\begin{itemize}
\item \textbf{State Growth Reduction:} Over 90\% reduction in active state growth rate
\item \textbf{Sync Time Improvement:} New validator synchronization in under one hour
\item \textbf{Hardware Requirements:} Support for consumer-grade hardware enabling 10,000+ independent validators
\item \textbf{Economic Sustainability:} Alignment of storage costs with actual usage patterns
\item \textbf{Backward Compatibility:} Seamless operation of existing contracts with transparent revival mechanisms
\end{itemize}

\section{Conclusion}

The PARADISE framework provides a comprehensive, production-ready solution to blockchain state bloat that preserves decentralization while ensuring long-term sustainability. Through the integration of state lifecycle management, stateless execution via the Continuous Proof Market, Verkle Tree data structures, and decentralized archival, we address the fundamental economic misalignment that drives unbounded state growth.

Our enhanced security design addresses critical vulnerabilities through reentrancy protection, anti-Sybil mechanisms, and MEV-resistant market structures. The phased migration strategy ensures smooth transition while maintaining backward compatibility.

The successful implementation of PARADISE will enable EVM-compatible networks to scale sustainably while preserving the core principles of decentralization and permissionless participation. This work represents a significant step toward resolving one of the most pressing challenges in blockchain infrastructure.

Future work will focus on formal verification of the cryptographic components, empirical validation through testnet deployment, and optimization of the economic parameters through mechanism design analysis.


\bibliographystyle{unsrt}
\bibliography{references}
\end{document}